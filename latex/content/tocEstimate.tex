%%%%%%%%%%%%%%%%%%%%%%%%%%%%%%%%%%%%%%%%%% Planned table of contents
% \IMRADlabel{introduction}
% \chapter{Introduction}

\section{Motivation}
Machine learning (ML) is a big research area in computer science and especially here at TU Darmstadt. Many groups in the computer science departments do research on ML either directly or indirectly by using it to achieve some other goal/knowledge (e.g. for image recognition or natural language processing). Research on computer languages concerning problems specific to ML could facilitate the implementation of ML applications for many use cases. Because of this the Software Technology Group at TU Darmstadt also wants to focus more on ML. Gradient descent and therefore differentiable programming builds the base of ML which makes it the perfect starting point for a newly found focus on ML. Having an overview of possible implementations would make it possible to decide which implementation is best suited to design a ML library, algorithm or even (domain specific) language around. 

Our goal for this work is to produce such an overview by implementing differentiation in multiple ways. These will be partly based on other's work which we (re-)implement and adjust to fit a similar implementation style so that we can easily compare all implementations. Furthermore we will extend those implementations with new functionalities, change them from the ground up or implement our own designs from scratch. Secondarily we will try to find use cases of the new language features introduced in Scala 3 to get an idea if this language has potential to aid implementing ML tasks.

\section{Overview}
Firstly we provide an overview over our work. We start off with implementing the normal well-known forward mode differentiation in \refsec{sec:forwardMode}. For this we first give a quick view on how to use a naive but unorthodox way to implement it using macros to replace code in \refsec{sec:macros}. After that we introduce the more common dual numbers in \refsec{sec:forwardDualNumbers} which will be a key ingredient also for later (reverse mode) implementations. The next implementation gets more experimental again. In \refsec{sec:matchTypes} we use the newly introduced match types and value types of Scala 3 to implement a symbolic differentiation-like way to produce a derivative completely at type level in compile time.

The main focus of this work is definitely \refsec{sec:reverseMode} because reverse mode is much more interesting for ML tasks than forward mode. Here we begin by introducing the mathematical background of reverse mode by doing it once ``by hand''. As using mutation and non-functional programming usually leads to an arguably more straight forward implementation we for now continue without restricting us to functional programming in \refsec{sec:mutation}. We introduce continuation passing style (CPS) which at first seems like unnecessary overhead and in itself may be hard to wrap your head around at first. However it pays off when it elegantly demystifies how to implement reverse mode differentiation in \refsec{sec:cps}. Using a tape to store operations on in \refsec{sec:tape} is also a valid way to implement reverse mode in a similar way to CPS. From there we continue to wrap CPS into a monad in \refsec{sec:monadCPS} to facilitate usage of CPS from the client side. We finish the mutation section by combining the monad and tape idea in \refsec{sec:monadAndTape}.

For the more puristic oriented in \refsec{sec:noMutation} we translate the CPS and monad implementations in \refsec{sec:functionalCps} and \refsec{sec:functionalMonad} respectively into a design which does use an accumulator to map expressions to their adjoints instead of mutation. The last implementation \refsec{sec:chad} is fully functional and doesn't even need a unique ID per expression which consequently means it only needs value equality to function. On top of that it is easily comprehensible because it comes without much overhead or complicated abstractions.

\todocite{section}
In the end we will compare the runtime performance of each implementation.

\section{External Work}
In conclusion we want to give a quick overview over which sections are based on which external work. \emph{CHAD: Combinatory Homomorphic Automatic Differentiation} \cite{chad} is the base for \refsec{sec:chad}. \emph{Demystifying Differentiable Programming: Shift/Reset the Penultimate Backpropagator} \cite{lantern} has a running example of a function which is differentiated. This function is also used throughout our work as a running example. Furthermore we based \refsec{sec:forwardDualNumbers}, \refsec{sec:cps}, \refsec{sec:tape} and \refsec{sec:functionalCps} directly on their work. Other reverse mode sections, namely \refsec{sec:monadCPS}, \refsec{sec:monadAndTape} and \refsec{sec:functionalMonad}, are mainly based on our own work but do refer to or build on concepts introduced in the previously mentioned sections and therefore are indirectly based on the work of Fei Wang et al. 



















% Hello
% For this work I need to cite \cite{abc12}. And \cite{tudapub}

% \section{bla}

% \subsection{blubb}

% Have a look at figure \ref{uml_example} to see how including images works.

% \begin{figure}
%     \centering
%     remember to include a path relative to your root .tex file
%     \includegraphics[width=4cm]{images/uml}
%     \caption{Sample UML Diagram}
%     \label{uml_example}
% \end{figure}

% \subsection{blubb 2}

% Avoid single subsections! Each ``parent'' has to have at least two ``childs''

% \section{bla 2}

\chapter{Introduction (2.5)}
	\section{Motivation, Why? -> ML needs this for Newton etc (0.5)}
% \IMRADlabel{methods}
	\section{Overview}
		\subsection{Short summary of all implementation styles (i.e. CPS, Monad, tape, macro) (0.5)}
	\section{Details}
		\subsection{Go through a real ML task (the implemented test) with "grad" as blackbox -> defines "goal" (1)}
		\subsection{Define "goal" (0.5)}
% % \chapter{Implementation}

\section{Forward Mode Differentiation}

Forward mode differentiation by hand is straight forward and also translates well into code by sticking to our existing knowledge about symbolic differentiation \todo{define symbolic differentiation?}. Essentially we want to go through every operation recursively and replace it with its derivative.


\subsection{Code replacement with macros}
If you take the last sentence literally you could now go the long and hard way (which we did), learn metaprogramming and implement it exactly by replacing expressions with their derivative. This approach looks rather ugly at first sight but after ignoring the boilerplate one can see that it just boils down to recursive pattern matching of code which works pretty well in Scala 3:
\begin{lstlisting}[language=scala]
    def d(t: Expr[Term])(using Quotes): Expr[Term] = t match
        case '{ ($l: Term) + ($r: Term) } =>    // l + r
            '{ ${ d(l) } + ${ d(r) } }          // d(l) + d(r)
        case '{ ($l: Term) * ($r: Term) } =>    // l * r
            '{ $l * ${ d(r) } + ${ d(l) } * $r }// l * d(r) + d(l) * d(r)
        case '{ X } => '{ 1 }                   // variable
        case '{ $v: V } => '{ 0 }               // constant
\end{lstlisting}
As this was just a small experiment we leave this here without much explanation for now. In fact we wouldn't even need macros and could just write our term with algebraic data types and recursively match them at runtime to implement this approach. This would reduce the boilerplate in comparison to the macro-implementation above.


\subsection{Operator overloading with dual numbers}
\todo{Following definitions, implementations by lantern paper (link to it or github)}
Rewriting code that way has one problem. We loose the actual result of our formula and only get the derivative (or we would have to calculate both separately). Instead we could use a structure that consists of two values, the normal result of the operation and its derivative. Such a construct is called a "dual number" \todo{link Wikipedia or maybe a paper?} and could look like this:
\lstscala{implementation/DualNumber.scala}{3}{3}
To implement an operation on these we have to define the actual computation and additionally how to compute the derivative:
\lstscala{implementation/DualNumber.scala}{3}{12}
As we can see this comes down to translating mathematic symbolic differentiation rules into code (lines 4 and 9). How to define constants and the variable which we differentiate our function with respect to \todogrammar also comes naturally:
\lstscala{implementation/DualNumber.scala}{14}{15}
Differentiation of a function $f$ at $x = 3$ could then look like this:
\lstscala{implementation/DualNumber.scala}{18}{24}


\subsection{Static typing taken too far (Or: Let the compiler code for you)}
To explain the following approach we first have to look at match types which are essentially functions but at type level. We put one type in, look at it and decide which type to ``return''. It gets clear with the following example taken from the Scala 3 docs \cite{matchTypesScala3}. You could get the element type of any list-like type:
\begin{lstlisting}[language=scala]
    type Elem[X] = X match
        case String => Char
        case Array[t] => t
        case Iterable[t] => t
\end{lstlisting}
By recursively matching types we can implement a differentiator purely at type level:
\begin{lstlisting}[language=scala]
    type D[T <: Term] <: Term = T match
        case l * r => 
            l * D[r] + D[l] * r
        case l + r => 
            D[l] + D[r]
        case X => 
            V[1]
        case V[_] => 
            V[0]
\end{lstlisting}
\lstscalainline{T} is just a type but represents a full calculation. Furthermore notice that the \lstscalainline{*} and \lstscalainline{+} operators seen above are infix types and not methods: 
\begin{lstlisting}[language=scala]
    type +[L <: Term, R <: Term] <: Term
    type *[L <: Term, R <: Term] <: Term
\end{lstlisting}
Line 7 and 9 still look suspicious. What are integers doing as a type parameter? These are compile time value types which are now supported in Scala 3:
\begin{lstlisting}[language=scala]
    type V[C <: Double | Int] <: Term
\end{lstlisting}
\lstscalainline{C} is a \lstscalainline{Double} or \lstscalainline{Int} singleton value type. By using \lstscalainline{scala.compiletime.constValue[C]} we can extract the singleton value from a singleton type. 

If we combine all functionalities described above we gain the possibility to define a function and derive its derivative entirely at type level:
\begin{lstlisting}[language=scala]
    type F = V[2.2] * X + X * X * X
    val df: D[F] = initFromType[D[F]]
    val result: Double = df(3)
\end{lstlisting}
Unfortunately computing a result at type level is impossible because we want to support decimal numbers. Type level calculation with integers on the other hand would be very possible and is included in Scala 3. To allow decimals we have to use a function \lstscalainline{initFromType} which recursively matches the provided type at compile time and constructs a function which just computes the result later at runtime. As if this init-function wasn't inelegant enough we faced another much bigger problem. The compile time when differentiating three multiplications in a row was already at about 20 s and 4 multiplications bumped it up to over 1 min (5 multiplications took too long to test). The solution to this problem was rather anticlimactic. By updating the Scala compiler to the newest version (3.0.2 as of now) compile times are now consistently at about 1 s independent of expression complexity. \todo{Gain of type safety?} \todo{Explain Peano numbers?}

\chapter{Reverse mode differentiation}\label{sec:reverseMode}
\todo{Whole chapter of Wikipedia (better citation)}

In contrast to forward mode differentiation which benefits highly from our intuition, reverse mode is not straight forward to implement and we even have to work constantly against our intuition. To make this difference clear let's go through an example taken from Wikipedia \cite{forwardAccumulationWiki}:
\newcommand{\yExampleDiff}{
    \begin{alignat*}{2}
        y & = x_1x_2 &  & + \sin{x_1} \\
          & = w_1w_2 &  & + \sin{w_1} \\
          & = w_3    &  & + w_4       \\
          & =        &  & w_5
    \end{alignat*}
}
\yExampleDiff
We gave every possible subexpression a name $w_i$. Note that the outmost expression has the largest index and the innermost expressions have the smallest indices. Order of indices at the same level do not matter.

In forward mode we calculate
\[ \dot w_i \coloneqq \diff{w_i}{x} \]
from small $i$ to largest. For example if we know $\dot w_1$ and $\dot w_2$ we can (by using the product rule) calculate
\[ \dot w_3 = w_1 \dot w_2 + \dot w_1 w_2. \]
With that (and $\dot w_4$) we can then find $\dot w_5$. Note that as stated above we need to know the initial value of $x_1$ and $x_2$. With one variable we would set it to $\dot x = \diff{x}{x} = 1$ and calculate our result. With two variables we have to do two passes through the whole calculation, one with $\dot x_1 = 1, \dot x_2 = 0$ and one with $\dot x_1 = 0, \dot x_2 = 1$. This is the whole motivation to do reverse mode differentiation because it only needs one pass to calculate the same result. In fact for a function $f: \R^n \to \R^m$ forward mode has to do $n$ passes and reverse mode has to do $m$ passes. Usually in machine learning tasks you have functions with $n >\! \!> m$.

\newcommand{\overw}[1]{\overline{w}_{#1}}
\newcommand{\diffyw}[1]{\diff{y}{w_{#1}}}
\todo{Cite Wikipedia reverse mode}
For reverse mode we have to shift our focus from $\dot w_i$ to another main expression of concern, namely
\[ \overw{i} \coloneqq \diffyw{i} \]
also called the adjoint. Instead of calculating the derivative of an subexpression $w_i$ with respect to $x$ it expresses the derivative of $y$ with respect to a particular subexpression $w_i$ of $y$. Why this expression concerns us now gets clear after looking at the corresponding usage of the chain rule \todo{Define chain rule?} which both differentiation styles are based on. Forward mode replaces all occurrences of $\dot w_i$ by using the chain rule which is the reason why that expression concerned us previously. The chain rule for backward mode is instead used to replace each occurrence of $\overw{i}$ recursively:
\newcommand{\diffw}[2]{\diff{w_{#1}}{w_{#2}}}
\[ \diff{y}{x} = \diffyw{1}\diff{w_1}{x} = \bigg(\diffyw{2}\diffw{2}{1}\bigg)\diff{w_1}{x} = \bigg(\bigg(\diffyw{3}\diffw{3}{2}\bigg)\diffw{2}{1}\bigg)\diff{w_1}{x} = ... \]
On first sight it might look like we traverse from $i = 1$ to $5$ but as one calculates the expression in the innermost parentheses first one can easily see that the iteration actually goes from large to small index opposed to forward mode \todogrammar.

This usage of the chain rule essentially dictates how we compute the reverse mode derivative. Our job is to calculate all $\overw{i}$ by applying the chain rule recursively until we have rewritten it to an expression including trivial subexpressions or $\overw{j}$ with $j > i$ which we have already computed. \todo{You could interpret that as applying the chain rule in reverse order i.e.} We use the same $y$ as above:
\yExampleDiff
\begin{align*}
    \overw{5}   & = \diffyw{5} = 1                                                                                                           \\
    \overw{4}   & = \diffyw{4} = \diffyw{5}\diffw{5}{4} = \overw{5}\diffw{5}{4} = 1                                                          \\
    \overw{3}   & = \diffyw{3} = \diffyw{5}\diffw{5}{3} = \overw{5}\diffw{5}{3} = 1                                                          \\
    \overw{2}   & = \diffyw{2} = \diffyw{5}\diffw{5}{2} = \bigg(\diffyw{5}\diffw{5}{3}\bigg)\diffw{3}{2} = \overw{3}\diffw{3}{2} = w_1       \\
    \intertext{These were straight forward after recognizing the general pattern and required simple usage of the chain rule. Calculating $w_1$ it not as straight forward:}
    \overw{1}^a & = \diffyw{1} = \diffyw{5}\diffw{5}{1} = \bigg(\diffyw{5}\diffw{5}{3}\bigg)\diffw{3}{1} = \overw{3}\diffw{3}{1} = w_2       \\
    \overw{1}^b & = \diffyw{1} = \diffyw{5}\diffw{5}{1} = \bigg(\diffyw{5}\diffw{5}{4}\bigg)\diffw{4}{1} = \overw{4}\diffw{4}{1} = \cos(w_1) \\
    \overw{1}   & = \overw{1}^a + \overw{1}^b
\end{align*}
We had to realise that $w_1$ appears in two ``calculation branches'' and had to handle them separately. Lastly the partial adjoints of all branches then have to be summed (implying that if $w_1$ would occur $n$ times we had to sum $n$ results) into the full adjoint.

The skilled reader could complain that this isn't as complicated as advertised \todowording earlier \todo{Did I?}. This is true for a human who can overlook the whole expression including all its subexpressions. A program on the other hand often only has a limited view of the whole expression. Consider this translation of our example into code:
\begin{lstlisting}
def w3 = w1 * w2
def w4 = sin(w1)
def y =  w3 + w4 // w5
\end{lstlisting}
At definition time of \lstinline{w3} we don't have enough information to calculate an andjoint because only after adding all context with the definition of \lstinline{y} we know all branches where \lstinline{w1} occurs. In fact \lstinline{y} could also be just another subexpression of a bigger calculation. Usually when evaluating expressions you can start evaluating the innermost subexpression and use its result to evaluate its containing expression as we did for forward mode differentiation. Unfortunately this isn't possible for reverse mode and directly using dual numbers as is doesn't suffice. We have to start with the top expressione $w_5$ and work our way down to (all occurrences of) $w_1$ (and $w_2$). This is very unnatural to implement because the information flows in reverse order (\reffig{fig:informationFlow}). \todo{Two arrows for reverse mode: values up, adjoints down}
\begin{center}
    % https://tikzcd.yichuanshen.de/#N4Igdg9gJgpgziAXAbVABwnAlgFyxMJZABgBoAmAXVJADcBDAGwFcYkQAdDuZgIzhz0AxgGtgAdwD6ARgAEXLrIAeMgL4hVpdJlz5CKchWp0mrdlx79BoiZPLyFyu+s3bseAkWmlpxhizZETm4+AWExKQBmB0UAKhctEAx3PSJInz9TQODLMJspABYYjgdsMAS3XU8DUmJMgPMQq3DbAFZi2QBqCqSdD31kdKoafzMgi1DrCJkOlWke5KqBgqMRrPYNRMX+ohW6tYagjWMYKABzeCJQADMAJwgAWyRDEBwIJG8QAAsYeih2SBgNg0QRYRgAghsVwgO6PZ4g96IMjfX7-IKA4EgXgwMBo5FwL5Ya44JDEaGwp6IFavREvH5-CFAzY3e6U6lvJDpFEM9GQ5kw1mchFIVo0elo8B88mCxAANmFiAA7DQCUSSYgXqNshNmjYuDgYEocMBrhBbuJ6LcoKoOvrDcaHtAYKp1CD6GDGVDEhSkMqaUh5SBVcSkABaUUmQ45SYtO1G4C3GC0GC3ODO20cA3xx2wF3HVRAA
    \begin{tikzcd}
        &                                                             & \substack{w_5 \\ +} \arrow[ld, no head] \arrow[rd, no head] &                                           & {} \arrow[dd, "\substack{\text{reverse} \\ \text{mode}}", shift left=5]  \\
        & \substack{w_3 \\ *} \arrow[rd, no head] \arrow[ld, no head] &                                                             & \substack{w_4 \\ \sin} \arrow[d, no head] &                                                                          \\
    \substack{w_1 \\ x_1} &                                                             & \substack{w_2 \\ x_2}                                       & \substack{w_1 \\ x_1}                     & {} \arrow[uu, "\substack{\text{forward} \\ \text{mode}}", shift right=2]
    \end{tikzcd}

    \captionsetup{type=figure}
    \caption{Information flow of forward and reverse mode}
	\label{fig:informationFlow}
\end{center}

The main takeaway from this elaborate \todowording example is an important pattern which we will be utilizing to implement reverse mode differentiation. The second last term of every calculation of $\overw{i}^z$ with $z \in \{a, b, ...\}$ \todo{explain z better?, we could also write $\overw{5} = \overw{5}^a = ...$ instead} has always the following pattern:
\newcommand{\defoverwiz}{\overw{i}^z = \overw{p}\diff{w_p}{w_i^z}}
\[ \defoverwiz \]
for some $p$. This $p$ (\emph{parent index}) isn't at all random. $w_p$ is always the parent expression of that specific occurrence $w_i^z$ of $w_i$. We use the superscript to distinguish specific occurrences. We also could have written $y$ like (notice the added superscripts $a$ and $b$)
\begin{alignat*}{2}
    y & = x_1x_2 &  & + \sin{x_1} \\
      & = w_1^a w_2^a &  & + \sin{w_1^b} \\
      & = w_3^a    &  & + w_4^a       \\
      & =        &  & w_5^a
\end{alignat*}
to further emphasize distinct occurrences of expressions $w_i$. Usually we omit the superscript if that expression only occurs once. Actually $w_i$ can only occur multiple times (i.e. having a ``$b$'' superscript) if $w_i = x_i$, i.e. it is one of our variables which we differentiate our function with respect to \todogrammar. In other words: Equal expressions are not counted as multiple occurrences and for that matter only $x_i$ count.

When boiling down
\[ \overw{i}^z = \overw{p}\diff{w_p}{w_i^z} = \diffyw{p}\diff{w_p}{w_i^z} \]
further we realize that we have to compute the derivative of the parent expression with respect to $w_i^z$. This is fortunately comparatively easy. A parent expression will always be an atomic operation (e.g. (*), (+), (-), $\sin$) and $w_i^z$ is always a direct argument. Because we usually know the derivative of each of our atomic operations, we can simply handle every case, e.g. for multiplication:
\[ \diff{w_p}{w_i^z} = \diff{(w_i^z \cdot w_k)}{w_i^z} = w_k \]
The remaining and main task is to find $\overw{p}$, \todopunctuation the adjoint of the parent expression (i.e. the derivative of $y$ with respect to $w_p$). This is a recursive problem but unfortunately in reverse order because information flows from outer expression to inner \todo{or inner to outer?, just remove?} as already illustrated in \reffig{fig:informationFlow}. Solving this ``reversed flow of information'' to calculate $\overw{p}$ elegantly, efficiently or easily to reason about is the main goal of the following implementations. \todo{define better goal according to goal of my theses?}


\section{Using mutation}
\section{Continuation Passing Style (CPS)}

\todocite{Lantern paper}

For our first reverse mode attempt \todowording we want to build on already implemented and understood code, i.e. our dual number implementation from \reflst{lst:lst:forwardDualNumber}. A very similar structure can be achieved by using continuation passing style (CPS). This is just a fancy term \todowording for the frequently used callbacks (e.g. for frontend web development) \todocite{?}. Essentially you pass the ``rest of the calculation'' to the function instead of using its return value and manually applying the ``rest'' on that result. To make things clear, \todopunctuation consider chaining two arbitrary functions (with unspecified types \lstinline{A, B, C, D}) as usual:
\todo{Add "composed" function of lines 11-13 (in both examples)}
\begin{lstlisting}
def first(x: A): B = ???
def second(x: B): D = ???

val a: A = ???
val firstResult: B = first(a)
val secondResult: D = second(firstResult)
\end{lstlisting}
And now an equal implementation but with continuations:
\begin{lstlisting}
def first[R](x: A)(rest: B => R): R = ???
def second[R](x: B)(rest: D => R): R = ???

val a: A = ???
val secondResult: D = 
    first(a) { firstResult => 
        second(firstResult) { identity }
    }
\end{lstlisting}
Notice that the input type of \lstinline{rest} in line 1 is \lstinline{B} which matches the result type of \lstinline{first} from the ordinary example above (and analogously for \lstinline{D} and \lstinline{second}). To show the general equality of both approaches we also introduced type parameter \lstinline{R} to both functions. This is needed to support arbitrary ``second'' functions even if they do not return exactly \lstinline{D}. In line 9 we pass \lstinline{identity} to \lstinline{second} to mark the ``end'' of the calculation. If we had a third operation, we would pass a further nested lambda which calls \lstinline{third} (and so on for \lstinline{fourth}, \lstinline{fifth}, ...). When following CPS strictly, every function takes a continuation. and ordinary variables are never used. Lambdas with named parameters fullfil that role instead (as seen with \lstinline{firstResult} above).
\todo{Add theoretical explanation of code above (\url{https://pbs.twimg.com/media/FAjvTXRXEAwKuT-?format=png&name=small})}

Using this, at first sight rather obscure, feature we can implement reverse mode very similar to \reflst{lst:lst:forwardDualNumber}:
\begin{lstlisting}[mathescape=true]
case class Dual(x: Double, var adjoint: Double):
    def *(that: Dual)(k: Dual => Dual): Dual =
        // Compute result of current operation and init adjoint with 0
        val localResult = Dual(this.x * that.x, 0) 

        // "wait" for all remaining (nested) calculations to finish which implicitly calculate the adjoint of the current operation (localResult.adjoint)
        val globalResult = k(localResult)

        // We can now compute the partial adjoint of the current computation branch for "this" and "that" respectively. If "this" or "that" occur in other computation branches then that brach is responsible for adding its partial adjoint.

        // += $\overw{\text{this}}^z$
        this.adjoint +=
            // $\overw{\text{this}}^z$
            that.x // $\diff{w_p}{w_{\text{this}}^z}$
                * localResult.adjoint // $\overw{p}$

        // += $\overw{\text{that}}^z$
        that.adjoint +=
            this.x // $\diff{w_p}{w_{\text{that}}^z}$
                * localResult.adjoint  // $\overw{p}$

        globalResult
    end *

    // Analogous to (*)
    def +(that: Dual)(k: Dual => Dual): Dual =
        val localResult = Dual(this.x + that.x, 0)
        val globalResult = k(localResult)
        this.adjoint +=
            1 // $\diff{w_p}{w_{\text{this}}^z}$
                * localResult.adjoint // $\overw{p}$
        that.adjoint +=
            1 // $\diff{w_p}{w_{\text{that}}^z}$
                * localResult.adjoint // $\overw{p}$
        globalResult
    end +
end Dual

def differentiate(f: Dual => (Dual => Dual) => Dual)(x: Double): Double = {
    val xDual = Dual(x, 0)

    // The result of f does not interest us directly. We only need its side effects
    f(xDual) { topExpression => {
        // We have to mark the top-most expression manually because our program would have no way to recognize it
        topExpression.adjoint = 1 // $\diff{y}{y}$
        topExpression
    }
    }
    xDual.adjoint // $\overline{x} = \diff{y}{x}$
}

def f(x: Dual)(k: Dual => Dual): Dual =
    // 2 * x + x * x * x
    (2 * x) { y1 =>
        (x * x) { y2 =>
            (y2 * x) { y3 =>
                (y1 + y3) { k }
            }
        }
    }
end f

val derivative: Double = differentiate(f)(3)

\end{lstlisting}


\subsection{Tape} \label{sec:tape}

The following implementation is very similar to CPS but instead of building a stack of calls implicitly by calling continuations we build that ``call stack'' manually. Remember that the only goal we achieved by using continuations was a two pass design which we used to do some operations (compute regular result) in normal order and some operations (compute adjoint) in reverse order through the expression tree. Another way to achieve this is to do the forward pass as usual but on the way additionally save all operations which have to be done in the reverse pass for later. When we have collected every operation we just execute them in ``reverse'' order:
\begin{lstlisting}[mathescape=true]
var tape: Unit => Unit = _ => ()

case class Dual(x: Double, var adjoint: Double):
    def *(that: Dual): Dual =
        val localResult = Dual(this.x * that.x, 0)

        def addPartialAdjoint(
            thisOrThat: Dual,
            derivativeWrtThisOrThat: Double
        ): Unit => Unit =
            _ =>
                val partialAdjoint = 
                    localResult.adjoint * derivativeWrtThisOrThat
                thisOrThat.adjoint += partialAdjoint
        end addPartialAdjoint

        tape = addPartialAdjoint(this, that.x) andThen tape
        tape = addPartialAdjoint(that, this.x) andThen tape

        localResult
    end *

    def +(that: Dual): Dual = ???
end Dual
\end{lstlisting}
In line 1 we define a mutable \lstinline{tape} which we use to store operations on. These operations can only produce side effects because the \lstinline{tape} has type \lstinline{Unit => Unit} which cannot take nor return anything meaningful. We initialize it with a no-op.
The first part of the multiplication (lines 5 to 15) which includes \lstinline{addPartialAdjoint} are in essence equal to the according lines in CPS, and therefore we will just highlight the differences. We also omitted the mathematical translations. They are still important to get the connection to the mathematical foundations but for them refer to \reflst{lst:cpsDual} as they are very similar.

First thing to note is the altered return type of \lstinline{addPartialAdjoint} (line 10). It now returns a function which in turn is just used for its side effects (\lstinline{Unit => Unit}). This means that when we call \lstinline{addPartialAdjoint} (lines 17 and 18) the adjoint is \emph{not} directly updated opposed to CPS. Instead, we prepend that ``operation'' (calculating and updating the adjoint of \lstinline{this} or \lstinline{that}) to \lstinline{tape}. We prepend (instead of appending) so that in the end we have a tape which executes each operation in reverse order of insertion.

Our \lstinline{differentiate} function is again similar to CPS:
\begin{lstlisting}
def differentiate(f: Dual => Dual)(x: Double): Double =
    tape = _ => ()
    val xDual: Dual = Dual(x, 0)
    val topExpression = f(xDual)
    topExpression.adjoint = 1
    tape(())
    xDual.adjoint
end differentiate

def f(x: Dual): Dual =
    Dual(2, 0) * x + x * x * x

val derivative = differentiate(f)(3)
\end{lstlisting}
This time \lstinline{differentiate} takes a simpler \lstinline{f} as its first argument because we do not use continuations anymore. It now just has an \lstinline{Dual} input and calculates a \lstinline{Dual}. Because \lstinline{tape} is a global variable we have to remember to reset it for every differentiation (line 2). We then call \lstinline{f} (line 4) to do the forward pass and to populate the \lstinline{tape}. Similar to CPS we have to manually set the adjoint of the top expression to $1 = \diff{y}{y}$ (line 5). At this point no differentiation has been done yet. We have to call \lstinline{tape} to start it manually (as it takes a \lstinline{Unit} we have to pass its only inhabitant, namely ``\lstinline{()}''). The definition of \lstinline{f} (line 10 and 11) is possibly the most interesting change. We do not need any continuations and can omit variable names which makes it easier to read and write.

To make the definition of \lstinline{f} even more regular we can define an implicit conversion which converts a constant into \lstinline{Dual} automatically. For this we use \lstinline{given} instances~\cite{givensScala3} of \lstinline{Conversion}~\cite{conversionsScala3} which were introduced in Scala 3. They specifically describe the intent to convert a value. Previously \lstinline{implicit} methods were used for this but their semantics were overloaded and for example have also been used to define extension methods. The first \lstinline{given} instance (line 1) is not needed for this example but is included for completeness if one uses decimal numbers:
\begin{lstlisting}
given Conversion[Double, Dual] = Dual(_, 0)
given Conversion[Int, Dual] = Dual(_, 0)

def f(x: Dual): Dual =
    // 2 is implicitly converted into Dual(2, 0)
    2 * x + x * x * x
\end{lstlisting}

So far we have only done reverse mode differentiation for one variable. As mentioned previously reverse mode differentiation shines when having multiple input variables. Therefore, it's apparent to make an example which supports that. Extending the tape implementation to take multiple variables is mostly trivial as we only have to change the \lstinline{differentiate} function:
\begin{lstlisting}
def differentiate(
    f: List[Dual] => Dual, 
    xs: List[Double]
): List[Double] =
    tape = _ => ()
    val xsDual: List[Dual] = xs map { Dual(_, 0) }
    f(xsDual).adjoint = 1
    tape(())
    xsDual map { _.adjoint }
end differentiate

def f(xs: List[Dual]): Dual =
    2 * xs(0) + xs(1) * xs(2) * xs(2)

val derivatives: List[Double] = differentiate(f, List(3.0, 5.0, 2.0))
\end{lstlisting}
We encode multiple variables as a single vector of type \lstinline{List[Dual]}. Because \lstinline{f} now takes a \lstinline{List[Dual]}, \lstinline{differentiate} has to reflect that by accepting a function \lstinline{List[Dual] => Dual} and a vector of values to differentiate \lstinline{f} at. At first, we have to reset the tape again (line 5). In line 6 we extract the \lstinline{adjoint} of each variable which ultimately gives us a vector where every value is the derivative with respect to one variable. In other words we computed the gradient of \lstinline{f}. The main takeaway here is that we computed the derivative of multiple variables in one go without having to call \lstinline{differentiate} multiple times with different values. This is only possible with reverse mode and is its main advantage. Forward mode would have to do one full differentiation for each variable where all other variable are set to 0.

Extending other reverse mode implementations for multidimensional functions (in input or output) is done analogously. To allow better focus on the essential differences and keep the examples simple we mostly concentrate on single dimensional functions from here on.

\section{Without mutation}
\subsection{Continuation Passing Style (CPS)}


\chapter{Forward Mode Differentiation [4] => +3}
	\section{Code replacement with macros [0.5]}
	\section{Operator overloading with dual numbers [1]}
	\section{Static typing taken too far (Or: Let the compiler code for you) [1.5] => +0.5}

\chapter{Reverse mode differentiation [10.5] + (11.5)}
	\section{[4] => +4}
	\section{Using mutation [6.5] + (3) => +3}
		\subsection{Continuation Passing Style (CPS) [4]}
		\subsection{Tape [2.5] => +2}
		\subsection{Monad (3)}
			\subsubsection{For-comprehensions (flatMap, map)}
			\subsubsection{First try: .map() "closes" monad}
			\subsubsection{Second try: Chaining of functions with global tape (translate from tape)}
	\section{Without mutation (8.5) => +5.5}
		\subsection{Why no mutation (state, multiple calculations, reasoning) (0.5)}
		\subsection{Continuation Passing Style (CPS) (1)}
		\subsection{Monad (2)}
		\subsection{Forless Monad (1)}
		\subsection{Tape (2)}
		\subsection{Chad (2)}
			\subsubsection{Multidim => Every class that is a Monoid is supported}

\chapter{Evaluation (1.5)}
	\section{List all comparison metrics (Runtime, LOC, compile time) (1)}
	\section{Interpretation (0.5)}

\chapter{Related work (0.5)}
	\section{lantern paper}
	\section{Microsoft paper (=> more efficiency)}

% \IMRADlabel{discussion}
\chapter{Conclusion (0.5)}

% Pages: 2.5 + 4 + 21.5 + 1.5 + 0.5 + 0.5 = 30
