\chapter{Reverse mode differentiation}\label{sec:reverseMode}

In contrast to forward mode differentiation which benefits highly from our intuition, reverse mode is not straight forward to implement and we even have to work constantly against our intuition. To make this difference clear let's go through the forward mode example from Wikipedia \cite{forwardAccumulationWiki}, using a similar notation and terminology:
\newcommand{\yExampleDiff}{
    \begin{alignat*}{2}
        y & = x_1x_2 &  & + \sin{x_1} \\
          & = w_1w_2 &  & + \sin{w_1} \\
          & = w_3    &  & + w_4       \\
          & =        &  & w_5
    \end{alignat*}
}
\yExampleDiff
For our purpose each expression is either an operation acting on subexpressions or a value (constant or variable). This can be conveniently visualized as an expression tree with operations as nodes and values as leaves:

\begin{minipage}[c]{0.5\textwidth}
    \begin{alignat*}{2}
        & =        &  & w_5 \\
        \\
        & = w_3    &  & + w_4       \\
        \\
        & = w_1w_2 &  & + \sin{w_1} \\
        \\
        y & = x_1x_2 &  & + \sin{x_1} \\
    \end{alignat*}
\end{minipage}%%
\begin{minipage}[c]{0.5\textwidth}
    % https://tikzcd.yichuanshen.de/#N4Igdg9gJgpgziAXAbVABwnAlgFyxMJZABgBoAmAXVJADcBDAGwFcYkQAdDuZgIzhz0AxgGtgAdwD6ARgAEXLrIAeMgL4hVpdJlz5CKchWp0mrdlx79BoiZPLyFyu+s3bseAkWmlpxhizZETm4+AWExKQBmB0UAKhctEAx3PSJInz9TQODLMJspABYYjgdsMAS3XU8DUmJMgPMQq3DbAFZi2QBqCqSdD31kdKoafzMgi1DrCJkOlWkXYxgoAHN4IlAAMwAnCABbJEMQHAgkbxAACxh6KHZIMDYaQSxGW4I2VxBtvYPHk8QyC5XG5BO4PEC8GBgYEAuDnLAbHBIYgfL77RAFX4-QHXV73DSJVFIDFHP7pbHA8BvfGbHZosnHJCtGiXHEgqmqSiqIA
%    \begin{tikzcd}
%        &                                                             & \substack{w_5 \\ +} \arrow[ld, no head] \arrow[rd, no head] &                                           \\
%        & \substack{w_3 \\ *} \arrow[rd, no head] \arrow[ld, no head] &                                                             & \substack{w_4 \\ \sin} \arrow[d, no head] \\
%    \substack{w_1 \\ x_1} &                                                             & \substack{w_2 \\ x_2}                                       & \substack{w_1 \\ x_1}
%    \end{tikzcd}


    % https://tikzcd.yichuanshen.de/#N4Igdg9gJgpgziAXAbVABwnAlgFyxMJZABgBoAmAXVJADcBDAGwFcYkQB3AfQEYQBfUuky58hFOQrU6TVu27kBQkBmx4CRHqR7SGLNohAAdI3GYAjODnoBjANbBuAZgAEJky4BU-JcLViiJ21dWQNjUwsrWwduABY3dwTsMB9BP1ENCVJiEP12EzNLa3tHLgBWBI8AalTlVQzxZCCqGj05Q24+NJURdUagp1z2kAAPXl8e-0zkSUHW0PYxxW76vqIyOZk8wzGu6RgoAHN4IlAAMwAnCABbJEkQHAgkLRAACxh6KHZIMDYaaywjG+BDY3UuNzu-yeiDIbw+X0MPz+IHMMDACNhcFeWDOOCQxDBV1uiFiUMhcM+wN+E3BxNJD2hQQpCPAIJpRKQTMeSDKNHelMRbMJEMQvIZSAAbHz4VTQcpac8yYgAOzSgWs6nC4mw7mIAAcapZSIElH4QA
    \begin{tikzcd}
        &                                                             & \substack{w_5 \\ +} \arrow[ld, no head] \arrow[rd, no head] &                                           \\
        & \substack{w_3 \\ *} \arrow[rd, no head] \arrow[ld, no head] &                                                             & \substack{w_4 \\ \sin} \arrow[d, no head] \\
        w_1 \arrow[d, no head] &                                                             & w_2 \arrow[d, no head]                                      & w_1 \arrow[d, no head]                    \\
        x_1                    &                                                             & x_2                                                         & x_1
    \end{tikzcd}
\end{minipage}
We gave every possible subexpression a name $w_i$:
\begin{align*}
    w_1 &\coloneqq x_1 \\
    w_2 &\coloneqq x_2 \\
    w_3 &\coloneqq w_1 w_2 \\
    w_4 &\coloneqq \sin{w_1} \\
    w_5 &\coloneqq w_3 + w_4
\end{align*}
Note that the outermost expression has the largest index and the innermost expressions have the smallest indices. Order of indices at the same level do not matter. Note that each occurrence of a $x_i$ gets the same name as can be seen with $x_1$ which both got the name $w_1$. This gets important later.

\newcommand{\overw}[1]{\overline{w}_{#1}}
\newcommand{\diffyw}[1]{\diff{y}{w_{#1}}}
Forward and reverse mode differentiation are built on two different main formulas of concern:
\\
\begin{minipage}[c]{0.5\textwidth}
    \begin{equation*}
        \text{Derivative:}\ \dot w_i \coloneqq \diff{w_i}{x}
    \end{equation*}
\end{minipage}%
\begin{minipage}[c]{0.5\textwidth}
    \begin{equation*}
       \text{Adjoint:}\ \overw{i} \coloneqq \diffyw{i}
    \end{equation*}
\end{minipage}
\\ \\
In forward mode we compute the derivative from small $i$ to largest, i.e.\ from the leaves to the root of the tree. For example if $\dot w_1 = \dot x_1$ and $\dot w_2 = \dot x_2$ are given we can calculate (by using the product rule)
\[ \dot w_3 = w_1 \dot w_2 + \dot w_1 w_2. \]
With that (and $\dot w_4$) we can then find $\dot w_5$. Note that as stated above we need to know the initial value of $\dot x_1$ and $\dot x_2$. If we had one single variable we would set it to $\dot x = \diff{x}{x} = 1$ and calculate our result. But as we have two variables we have to do two passes through the whole calculation, one with $\dot x_1 = 1, \dot x_2 = 0$ and one with $\dot x_1 = 0, \dot x_2 = 1$. Reverse mode does not have to do this. It only has to do one pass to calculate the same result which is the whole motivation to do reverse mode instead of forward mode. In fact for a function $f: \R^n \to \R^m$ forward mode has to do $n$ passes and reverse mode has to do $m$ passes through the whole function. Usually in machine learning tasks you have functions with $n >\! \!> m$ which are also very complex. You certainly want to do as least passes as possible.

For the following parts we orient ourselves by the reverse mode example also from Wikipedia \cite{reverseAccumulationWiki}, again using similar notation and terminology. For reverse mode we have to shift our focus from $\dot w_i$ to another main expression of concern, namely
\[ \overw{i} \coloneqq \diffyw{i} \]
also called the adjoint of $w_i$. Instead of calculating the derivative of a subexpression $w_i$ with respect to $x$ it expresses the derivative of $y$ with respect to a particular subexpression $w_i$ of $y$. Why this expression concerns us now gets clear after looking at the corresponding usage of the chain rule which both differentiation styles are based on. A quick reminder on the definition of the chain rule:
\[ \diff{y}{x} = \diff{y}{z}\diff{z}{x} \]
Forward mode replaces all occurrences of $\dot w_i$ by using the chain rule which is the reason why that expression concerned us previously. The chain rule for backward mode is instead used to replace each occurrence of $\overw{i}$ recursively:
\newcommand{\diffw}[2]{\diff{w_{#1}}{w_{#2}}}
\[ \diff{y}{x} = \diffyw{1}\diff{w_1}{x} = \bigg(\diffyw{2}\diffw{2}{1}\bigg)\diff{w_1}{x} = \bigg(\bigg(\diffyw{3}\diffw{3}{2}\bigg)\diffw{2}{1}\bigg)\diff{w_1}{x} = ... \]
On first sight it might look like we traverse from $i = 1$ to $5$. However as one calculates the expression in the innermost parentheses first one can easily see that the iteration actually goes from large to small index opposed to forward mode.

This usage of the chain rule essentially dictates how we compute the reverse mode derivative. Our job is to calculate all $\overw{i}$ by applying the chain rule recursively until we have rewritten it to an expression including trivial subexpressions or $\overw{j}$ with $j > i$ which we have already computed. We use the same $y$ as above and calculate the adjoints of subexpressions $w_5$ to $w_2$:
\yExampleDiff
\begin{align*}
    \overw{5}   & = \diffyw{5} = 1                                                                                                           \\
    \overw{4}   & = \diffyw{4} = \diffyw{5}\diffw{5}{4} = \overw{5}\diffw{5}{4} = 1                                                          \\
    \overw{3}   & = \diffyw{3} = \diffyw{5}\diffw{5}{3} = \overw{5}\diffw{5}{3} = 1                                                          \\
    \overw{2}   & = \diffyw{2} = \diffyw{5}\diffw{5}{2} = \bigg(\diffyw{5}\diffw{5}{3}\bigg)\diffw{3}{2} = \overw{3}\diffw{3}{2} = w_1       \\
    \intertext{These were straight forward after recognizing the general pattern and required simple usage of the chain rule. Calculating $\overw{1}$ it not as straight forward:}
    \overw{1}^a & = \diffyw{1} = \diffyw{5}\diffw{5}{1} = \bigg(\diffyw{5}\diffw{5}{3}\bigg)\diffw{3}{1} = \overw{3}\diffw{3}{1} = w_2       \\
    \overw{1}^b & = \diffyw{1} = \diffyw{5}\diffw{5}{1} = \bigg(\diffyw{5}\diffw{5}{4}\bigg)\diffw{4}{1} = \overw{4}\diffw{4}{1} = \cos(w_1) \\
    \overw{1}   & = \overw{1}^a + \overw{1}^b
\end{align*}
We had to realise that $w_1$ appears in two ``calculation branches'' and had to handle them separately. Lastly these partial adjoints (i.e. $\overw{1}^a$ and $\overw{1}^b$) of all branches then have to be summed (implying that if $w_1$ would occur $n$ times we had to sum $n$ results) into the full adjoint $\overw{1}$.

After some consideration and breaking down every step this process is not very complicated. This is true for a human who can overlook the whole expression including all its subexpressions. A program on the other hand often only has a limited view of the whole expression. Consider this translation of our example into code:
\begin{lstlisting}
def w3 = w1 * w2
def w4 = sin(w1)
def y  = w3 + w4 // w5
\end{lstlisting}
At definition time of \lstinline{w3} we do not have enough information to calculate a full andjoint because only after adding all context with the definition of \lstinline{y} we know all branches where \lstinline{w1} occurs. In fact \lstinline{y} could also be just another subexpression of a bigger calculation. Usually when evaluating expressions you can start evaluating the innermost subexpression and use its result to evaluate its containing expression as we did for forward mode differentiation. This had the advantage that we could calculate the normal result values and the derivative of each subexpression simultaniously. Unfortunately this is not possible for reverse mode and directly using dual numbers as is does not suffice. We have to start with the top expressione $w_5$ and work our way down to (all occurrences of) $w_1$ (and $w_2$). This is very unnatural to implement because the information flows in reverse order. On top of that when iterating from outer to inner expression we can no longer calculate the normal result. Naturally at some point we have to calculate these values too. The only option we have is to do an full \emph{forward pass} (inner to outer) to calculate the result values and another full \emph{backward pass} (outer to inner) to calculate the (partial) adjoints of every subexpression (also see \reffig{fig:informationFlow}). Fortunately the forward pass is trivial as it only has to calculate arithmetic results in usual recursive order.

\begin{center}
    % https://tikzcd.yichuanshen.de/#N4Igdg9gJgpgziAXAbVABwnAlgFyxMJZARgBoAmAXVJADcBDAGwFcYkQAdDuZgIzhz0AxgGtgAdwD6xAARcuMgB7SAviBWl0mXPkIoAzBWp0mrdlx79BoiZPJz5Su2o1bseAkXKlixhizZETm4+AWExKX0HBQAqF00QDHddIgAWHz9TQODLMJspVOiOB2wweLcdTwNSAAZMgPMQq3DbAFYimQBqcsTtDz1kdKoafzMgi1DrCOkO5WIepMqB1qMRrPZ1BMX+ohW6tYbxjhwYRRxgACcYWhgLuBgZAFtoGAW+lJQa1ZNDkE2KnafWr1MbBE5nYAAMwgF3E9AuUCeLxcxhgUAA5vAiKBIRcII8kN4QDgIEgyCAABYwehQdiQMBsGiCLCMOkENiuEC4-GEpmkxBfSnU2lBemMkC8GBgEWCuAUrCQnBIGqc7kExDpYn8olUmlshn-Ll49WaklIQxCvWi9mGtXmvlIFaWkXgG2q41IABsDsQAHYaHKFUrEABaImjbITZo2Ljg87Q2HwqAqDqx07nND0OBwFPyYppiFCfFoZgnHOp47p4ARnNqJn0Fn6jkJO1+n3ekCBxVIEPkiONXJTYAF85XG53V4VuPATPZ3OOEfAIuPEtl+cKRc0gBWECwYBwtdtHsQAA4fQBOGiMeiSxgABXeVRAFyw6IpSoD8u7AoOoKjeTERdl1XeB13zSsIRrMCHGnehpWgxdYBfBg8BuQ8VEoFQgA
\begin{tikzcd}
    \text{forward mode}                                                                               &                       &                                                             & \substack{w_5 \\ +} \arrow[ld, no head] \arrow[rd, no head] &                                           & \text{reverse mode} \arrow[dd, "\substack{\text{reverse} \\ \text{pass} \\ \text{computes} \\ \text{adjoints}}", shift left] \\
                                                                                                      &                       & \substack{w_3 \\ *} \arrow[rd, no head] \arrow[ld, no head] &                                                             & \substack{w_4 \\ \sin} \arrow[d, no head] &                                                                                                                              \\
    {} \arrow[uu, "\substack{\text{computes} \\ \text{values} \\ \text{and} \\ \text{derivatives}}"'] & \substack{w_1 \\ x_1} &                                                             & \substack{w_2 \\ x_2}                                       & \substack{w_1 \\ x_1}                     & {} \arrow[uu, "\substack{\text{forward} \\ \text{pass} \\ \text{computes} \\ \text{values}}", shift left=2]                 
    \end{tikzcd}

    \captionsetup{type=figure}
    \caption{Information flow of forward and reverse mode}
	\label{fig:informationFlow}
\end{center}

The main takeaway from this extensive example is an important pattern which we will be utilizing to implement the reverse pass. The second last term of every calculation of $\overw{i}^z$ (i.e. the partial adjoint of the $z$-th occurrence of $w_i$) with $z \in \{a, b, ...\}$ has always the following pattern:
\newcommand{\defoverwiz}{\overw{i}^z = \overw{p}\diff{w_p}{w_i^z}}
\[ \defoverwiz \]
for some $p$. This $p$ (\emph{parent index}) is not at all random. $w_p$ is always the parent expression of that specific occurrence $w_i^z$ of $w_i$. We use the superscript to distinguish specific occurrences. We also could have written $y$ like (notice the added superscripts $a$ and $b$)
\begin{alignat*}{2}
    y & = x_1x_2 &  & + \sin{x_1} \\
      & = w_1^a w_2^a &  & + \sin{w_1^b} \\
      & = w_3^a    &  & + w_4^a       \\
      & =        &  & w_5^a
\end{alignat*}
to further emphasize distinct occurrences of expressions $w_i$. Usually we omit the superscript if that expression only occurs once. Actually $w_i$ can only occur multiple times (i.e. having a ``$b$'' superscript) if $w_i = x_i$, i.e. it is one of our variables (i.e. in this case either $x_1$ or $x_2$). In other words: Equal expressions are not counted as multiple occurrences and for that matter only $x_i$ count.

When boiling down
\[ \overw{i}^z = \overw{p}\diff{w_p}{w_i^z} = \diffyw{p}\diff{w_p}{w_i^z} \]
further we realize that we have to compute the derivative of the parent expression with respect to $w_i^z$. This is fortunately comparatively easy. A parent expression will always be an atomic operation (e.g. (*), (+), (-), $\sin$) and $w_i^z$ is always a direct argument. Because we usually know the derivative of each of our atomic operations, we can simply handle every case, e.g. for multiplication:
\[ \diff{w_p}{w_i^z} = \diff{(w_i^z \cdot w_k)}{w_i^z} = w_k \]
The remaining and main task is to find $\overw{p}$, the adjoint of the parent expression (i.e. the derivative of $y$ with respect to $w_p$). This is a recursive problem but unfortunately in reverse order because information flows from outer expression to inner which is the reason why this is called the reverse pass as already illustrated in \reffig{fig:informationFlow}. Solving this ``reversed flow of information'' to calculate $\overw{p}$ elegantly, efficiently or easily to reason about is the main goal of the following implementations.


\section{Using mutation} \label{sec:mutation}
\section{Continuation Passing Style (CPS)}

\todocite{Lantern paper}

For our first reverse mode attempt \todowording we want to build on already implemented and understood code, i.e. our dual number implementation from \reflst{lst:lst:forwardDualNumber}. A very similar structure can be achieved by using continuation passing style (CPS). This is just a fancy term \todowording for the frequently used callbacks (e.g. for frontend web development) \todocite{?}. Essentially you pass the ``rest of the calculation'' to the function instead of using its return value and manually applying the ``rest'' on that result. To make things clear, \todopunctuation consider chaining two arbitrary functions (with unspecified types \lstinline{A, B, C, D}) as usual:
\todo{Add "composed" function of lines 11-13 (in both examples)}
\begin{lstlisting}
def first(x: A): B = ???
def second(x: B): D = ???

val a: A = ???
val firstResult: B = first(a)
val secondResult: D = second(firstResult)
\end{lstlisting}
And now an equal implementation but with continuations:
\begin{lstlisting}
def first[R](x: A)(rest: B => R): R = ???
def second[R](x: B)(rest: D => R): R = ???

val a: A = ???
val secondResult: D = 
    first(a) { firstResult => 
        second(firstResult) { identity }
    }
\end{lstlisting}
Notice that the input type of \lstinline{rest} in line 1 is \lstinline{B} which matches the result type of \lstinline{first} from the ordinary example above (and analogously for \lstinline{D} and \lstinline{second}). To show the general equality of both approaches we also introduced type parameter \lstinline{R} to both functions. This is needed to support arbitrary ``second'' functions even if they do not return exactly \lstinline{D}. In line 9 we pass \lstinline{identity} to \lstinline{second} to mark the ``end'' of the calculation. If we had a third operation, we would pass a further nested lambda which calls \lstinline{third} (and so on for \lstinline{fourth}, \lstinline{fifth}, ...). When following CPS strictly, every function takes a continuation. and ordinary variables are never used. Lambdas with named parameters fullfil that role instead (as seen with \lstinline{firstResult} above).
\todo{Add theoretical explanation of code above (\url{https://pbs.twimg.com/media/FAjvTXRXEAwKuT-?format=png&name=small})}

Using this, at first sight rather obscure, feature we can implement reverse mode very similar to \reflst{lst:lst:forwardDualNumber}:
\begin{lstlisting}[mathescape=true]
case class Dual(x: Double, var adjoint: Double):
    def *(that: Dual)(k: Dual => Dual): Dual =
        // Compute result of current operation and init adjoint with 0
        val localResult = Dual(this.x * that.x, 0) 

        // "wait" for all remaining (nested) calculations to finish which implicitly calculate the adjoint of the current operation (localResult.adjoint)
        val globalResult = k(localResult)

        // We can now compute the partial adjoint of the current computation branch for "this" and "that" respectively. If "this" or "that" occur in other computation branches then that brach is responsible for adding its partial adjoint.

        // += $\overw{\text{this}}^z$
        this.adjoint +=
            // $\overw{\text{this}}^z$
            that.x // $\diff{w_p}{w_{\text{this}}^z}$
                * localResult.adjoint // $\overw{p}$

        // += $\overw{\text{that}}^z$
        that.adjoint +=
            this.x // $\diff{w_p}{w_{\text{that}}^z}$
                * localResult.adjoint  // $\overw{p}$

        globalResult
    end *

    // Analogous to (*)
    def +(that: Dual)(k: Dual => Dual): Dual =
        val localResult = Dual(this.x + that.x, 0)
        val globalResult = k(localResult)
        this.adjoint +=
            1 // $\diff{w_p}{w_{\text{this}}^z}$
                * localResult.adjoint // $\overw{p}$
        that.adjoint +=
            1 // $\diff{w_p}{w_{\text{that}}^z}$
                * localResult.adjoint // $\overw{p}$
        globalResult
    end +
end Dual

def differentiate(f: Dual => (Dual => Dual) => Dual)(x: Double): Double = {
    val xDual = Dual(x, 0)

    // The result of f does not interest us directly. We only need its side effects
    f(xDual) { topExpression => {
        // We have to mark the top-most expression manually because our program would have no way to recognize it
        topExpression.adjoint = 1 // $\diff{y}{y}$
        topExpression
    }
    }
    xDual.adjoint // $\overline{x} = \diff{y}{x}$
}

def f(x: Dual)(k: Dual => Dual): Dual =
    // 2 * x + x * x * x
    (2 * x) { y1 =>
        (x * x) { y2 =>
            (y2 * x) { y3 =>
                (y1 + y3) { k }
            }
        }
    }
end f

val derivative: Double = differentiate(f)(3)

\end{lstlisting}


\subsection{Tape} \label{sec:tape}

The following implementation is very similar to CPS but instead of building a stack of calls implicitly by calling continuations we build that ``call stack'' manually. Remember that the only goal we achieved by using continuations was a two pass design which we used to do some operations (compute regular result) in normal order and some operations (compute adjoint) in reverse order through the expression tree. Another way to achieve this is to do the forward pass as usual but on the way additionally save all operations which have to be done in the reverse pass for later. When we have collected every operation we just execute them in ``reverse'' order:
\begin{lstlisting}[mathescape=true]
var tape: Unit => Unit = _ => ()

case class Dual(x: Double, var adjoint: Double):
    def *(that: Dual): Dual =
        val localResult = Dual(this.x * that.x, 0)

        def addPartialAdjoint(
            thisOrThat: Dual,
            derivativeWrtThisOrThat: Double
        ): Unit => Unit =
            _ =>
                val partialAdjoint = 
                    localResult.adjoint * derivativeWrtThisOrThat
                thisOrThat.adjoint += partialAdjoint
        end addPartialAdjoint

        tape = addPartialAdjoint(this, that.x) andThen tape
        tape = addPartialAdjoint(that, this.x) andThen tape

        localResult
    end *

    def +(that: Dual): Dual = ???
end Dual
\end{lstlisting}
In line 1 we define a mutable \lstinline{tape} which we use to store operations on. These operations can only produce side effects because the \lstinline{tape} has type \lstinline{Unit => Unit} which cannot take nor return anything meaningful. We initialize it with a no-op.
The first part of the multiplication (lines 5 to 15) which includes \lstinline{addPartialAdjoint} are in essence equal to the according lines in CPS, and therefore we will just highlight the differences. We also omitted the mathematical translations. They are still important to get the connection to the mathematical foundations but for them refer to \reflst{lst:cpsDual} as they are very similar.

First thing to note is the altered return type of \lstinline{addPartialAdjoint} (line 10). It now returns a function which in turn is just used for its side effects (\lstinline{Unit => Unit}). This means that when we call \lstinline{addPartialAdjoint} (lines 17 and 18) the adjoint is \emph{not} directly updated opposed to CPS. Instead, we prepend that ``operation'' (calculating and updating the adjoint of \lstinline{this} or \lstinline{that}) to \lstinline{tape}. We prepend (instead of appending) so that in the end we have a tape which executes each operation in reverse order of insertion.

Our \lstinline{differentiate} function is again similar to CPS:
\begin{lstlisting}
def differentiate(f: Dual => Dual)(x: Double): Double =
    tape = _ => ()
    val xDual: Dual = Dual(x, 0)
    val topExpression = f(xDual)
    topExpression.adjoint = 1
    tape(())
    xDual.adjoint
end differentiate

def f(x: Dual): Dual =
    Dual(2, 0) * x + x * x * x

val derivative = differentiate(f)(3)
\end{lstlisting}
This time \lstinline{differentiate} takes a simpler \lstinline{f} as its first argument because we do not use continuations anymore. It now just has an \lstinline{Dual} input and calculates a \lstinline{Dual}. Because \lstinline{tape} is a global variable we have to remember to reset it for every differentiation (line 2). We then call \lstinline{f} (line 4) to do the forward pass and to populate the \lstinline{tape}. Similar to CPS we have to manually set the adjoint of the top expression to $1 = \diff{y}{y}$ (line 5). At this point no differentiation has been done yet. We have to call \lstinline{tape} to start it manually (as it takes a \lstinline{Unit} we have to pass its only inhabitant, namely ``\lstinline{()}''). The definition of \lstinline{f} (line 10 and 11) is possibly the most interesting change. We do not need any continuations and can omit variable names which makes it easier to read and write.

To make the definition of \lstinline{f} even more regular we can define an implicit conversion which converts a constant into \lstinline{Dual} automatically. For this we use \lstinline{given} instances~\cite{givensScala3} of \lstinline{Conversion}~\cite{conversionsScala3} which were introduced in Scala 3. They specifically describe the intent to convert a value. Previously \lstinline{implicit} methods were used for this but their semantics were overloaded and for example have also been used to define extension methods. The first \lstinline{given} instance (line 1) is not needed for this example but is included for completeness if one uses decimal numbers:
\begin{lstlisting}
given Conversion[Double, Dual] = Dual(_, 0)
given Conversion[Int, Dual] = Dual(_, 0)

def f(x: Dual): Dual =
    // 2 is implicitly converted into Dual(2, 0)
    2 * x + x * x * x
\end{lstlisting}

So far we have only done reverse mode differentiation for one variable. As mentioned previously reverse mode differentiation shines when having multiple input variables. Therefore, it's apparent to make an example which supports that. Extending the tape implementation to take multiple variables is mostly trivial as we only have to change the \lstinline{differentiate} function:
\begin{lstlisting}
def differentiate(
    f: List[Dual] => Dual, 
    xs: List[Double]
): List[Double] =
    tape = _ => ()
    val xsDual: List[Dual] = xs map { Dual(_, 0) }
    f(xsDual).adjoint = 1
    tape(())
    xsDual map { _.adjoint }
end differentiate

def f(xs: List[Dual]): Dual =
    2 * xs(0) + xs(1) * xs(2) * xs(2)

val derivatives: List[Double] = differentiate(f, List(3.0, 5.0, 2.0))
\end{lstlisting}
We encode multiple variables as a single vector of type \lstinline{List[Dual]}. Because \lstinline{f} now takes a \lstinline{List[Dual]}, \lstinline{differentiate} has to reflect that by accepting a function \lstinline{List[Dual] => Dual} and a vector of values to differentiate \lstinline{f} at. At first, we have to reset the tape again (line 5). In line 6 we extract the \lstinline{adjoint} of each variable which ultimately gives us a vector where every value is the derivative with respect to one variable. In other words we computed the gradient of \lstinline{f}. The main takeaway here is that we computed the derivative of multiple variables in one go without having to call \lstinline{differentiate} multiple times with different values. This is only possible with reverse mode and is its main advantage. Forward mode would have to do one full differentiation for each variable where all other variable are set to 0.

Extending other reverse mode implementations for multidimensional functions (in input or output) is done analogously. To allow better focus on the essential differences and keep the examples simple we mostly concentrate on single dimensional functions from here on.
\subsection{Monad CPS} \label{sec:monadCPS}

We already established that writing CPS-functions by hand is cumbersome and not easily readable. We don't want to write deeply nested lambdas only to represent simple arithmetic calculations. Optimally we want to write arithmetic functions without a syntactic constraint, for example like this:
\begin{lstlisting}
2 * x + x * x * x
\end{lstlisting}
Because this isn't translatable into CPS easily, the next best thing would be a syntax which resembles common usage of Scala to utilize our inherent intuition instead of breaking it. We introduce a named value for every subexpression. This doesn't change any semantics but conveniently separates each subexpression into its own line and own syntactic construct:
\begin{lstlisting}
val y1 = x * 2
val y2 = x * x
val y3 = y2 * x
val y4 = y1 + y3
\end{lstlisting}
Admittedly introducing a mandatory value name for every subexpression isn't as elegant as directly writing down the expression. But value definitions are so ubiquitous that writing and reading them is at least very intuitive. The syntax should therefore look similar to this. Notice that we have only ``defined'' how we would like our code to look like and haven't solved our problem yet as continuations are nowhere to be found yet. However we are not as far away from CPS as one could think. Remember how every continuation also has a mandatory named parameter which represents the result of the last calculation. By introducing mandatory named value definitions we have a somewhat similar situation at hand. Our goal is now to automatically rewrite these imperative value definitions into a CPS construct where each \lstinline{val} is translated into a continuation parameter and each subexpression is nested into the continuation of the last one. 

Turns out Scala's for-comprehensions, if used in a specific way, can do exactly that. We mainly make use of the fact that a for-comprehension (without any guards) is entirely desugared into calls of multiple nested \lstinline{flatMap}s and one concluding \lstinline{map} for the \lstinline{yield}. If we manage to implement those two methods for our dual numbers, which is essentially equivalent to implementing a monad, we can write a function like this:
\begin{lstlisting}
def f(x: Dual): DualMonad
    for
        y1 <- x * 2
        y2 <- x * x
        y3 <- y2 * x
        y4 <- y1 + y3
    yield y4    
\end{lstlisting}
\lstinline{f} now returns a \lstinline{Monad} which wraps a dual number. Except of changed syntax the code is essentially similar to the imperative code of the last listing. The compiler then desugars it into this:
\begin{lstlisting}[caption={Desugared for-comprehension}, label={lst:desugaredForComprehension}]
def f(x: Dual): DualMonad =
    (x * 2).flatMap { y1 =>
        (x * x).flatMap { y2 =>
            (y2 * x).flatMap { y3 =>
                (y1 + y3).map { y4 =>
                    y4
                }
            }
        }
    }
end f  
\end{lstlisting}
At this point it should get clear why we wanted to use for-comprehensions to abstract over CPS. The compiler does the hard work for us and almost exactly translates a for-comprehension into CPS. Implementing \lstinline{flatMap} and \lstinline{map} (and thereby a monad) is the last (and main) task. The remaining code structure matches our previous CPS implementation.

Let's first look at the desired general signature for \lstinline{flatMap} and \lstinline{map} of a general monad:
\begin{lstlisting}
trait Monad[A]:
    def flatMap[B](f: A => Monad[B]): Monad[B] = ???
    def map[B](f: A => B): Monad[B] = ???
\end{lstlisting}
\lstinline{A} represents the value we are wrapping with the monad and \lstinline{B} is an arbitrary new type (possibly same as \lstinline{A}) which the value of \lstinline{A} is converted to. Both methods return a new monad which now wraps \lstinline{B}. In essence both methods model the mutation of the wrapped value and allow for the value type to change. The difference lies in the function passed to them. While the passed function to \lstinline{flatMap} returns a monad, the passed function to \lstinline{map} only returns a new value and \lstinline{map} has to wrap \lstinline{B} itself to be ultimately able to return \lstinline{Monad[B]}. 

A monad defined like this is very versatile because all value types are parameterized. This is important when using advanced capabilities of monads and to understand the concept in itself. For our use case on the other hand this is clearly excessive and can be simplified. The only value type we work on is \lstinline{Dual} and therefore we can replace all occurrences of \lstinline{A} and \lstinline{B} with simply \lstinline{Dual}. \lstinline{DualMonad} can be simply interpreted like an alias for \lstinline{Monad[Dual]}. For our purposes this is enough and makes the code easier to read without sacrificing expressiveness:
\begin{lstlisting}
trait DualMonad:
    def flatMap(k: Dual => DualMonad): DualMonad = ???
    def map(k: Dual => Dual): DualMonad = ???
\end{lstlisting}
We also renamed the passed functions into \lstinline{k} because they clearly represent continuations like one can easily see when looking at the desugared for-comprehension in \reflst{lst:desugaredForComprehension}. Let's look at the full implementation of \lstinline{Dual} and how we could implement \lstinline{DualMonad}:
\begin{lstlisting}
case class Dual(x: Double, var adjoint: Double):
    thisDual =>
  
    def *(thatDual: Dual): DualMonad = new DualMonad {
      override def flatMap(k: Dual => DualMonad): DualMonad =
        val parent = Dual(thisDual.x * thatDual.x, 0)
        val result = k(parent)
        thisDual.adjoint += thatDual.x * parent.adjoint
        thatDual.adjoint += thisDual.x * parent.adjoint
        result
  
      override def map(k: Dual => Dual): DualMonad =
        def wrap(dual: Dual): DualMonad = dual * 1
        flatMap(k andThen wrap)
    }

    def +(r: Dual): DualMonad = ???
end Dual
\end{lstlisting}
First notice that we don't implement \lstinline{DualMonad} at top level and in fact only implement it ad hoc when calling an operation on \lstinline{Dual}s. We use this to override \lstinline{flatMap} with the main operation logic which was found directly as part of \lstinline{*} (or \lstinline{+}) in previous reverse mode implementations. Because multiplication and addition have different logic we implement them ad hoc. When inspecting \lstinline{flatMap} further we realize that it is almost equivalent to the code of \lstinline{*} from the CPS implementation in \reflst{lst:cpsDual}. We calculate the parent result (line 6), call the continuation k to build the ``call stack'' (line 7) and then use the reverse pass to update the adjoints of \lstinline{thisDual} and \lstinline{thatDual} using the adjoint of the parent expression (lines 8-9). Note that we appended -\lstinline{Dual} to the instance name in line 2 to prevent a name clash with identifier \lstinline{this} when we are inside the ad hoc definition of \lstinline{DualMonad}. The only but important difference is that we eventually return the \lstinline{DualMonad} we got from the continuation in line 7. Returning a \lstinline{DualMonad} allows us to chain multiple \lstinline{flatMap} calls together which in turn means chaining multiple operations just like with normal CPS.

The next method we override is \lstinline{map} (line 12). The important signature difference to \lstinline{flatMap} is that it gets a function which returns a \lstinline{Dual} directly instead of a wrapped \lstinline{DualMonad}. Usually this is used to allow a last operation directly on \lstinline{Dual} before ending the for-comprehension. In our case this is not needed because each of our operations on \lstinline{Dual} return \lstinline{DualMonad} and not \lstinline{Dual}. Because of this the only \lstinline{k} which can be passed is a function which returns its argument or another \lstinline{Dual} captured by its closure. Essentially our job is to wrap the result of \lstinline{k} into a monad without changing the semantics. Naively one would try to implement \lstinline{DualMonad} ad hoc as we did with \lstinline{flatMap}. One unfortunately quickly realizes that we would have to implement another \lstinline{DualMonad} in the nested \lstinline{map} which leads us into an infinite loop of implementations. A solution for this is to apply a trivial identity function like multiplying with 1 to wrap a \lstinline{Dual} into a \lstinline{DualMonad} (line 13). With the ability to wrap a \lstinline{Dual} we can directly use the already implemented \lstinline{flatMap}. We just pass \lstinline{k} to it but wrap \lstinline{k}'s result into a \lstinline{DualMonad} to conform to \lstinline{flatMap}'s signature.

We successfully implemented a Monad and now just need a \lstinline{differentiate} function which sets the adjoint of the top expression to 1 just like in previous implementations. Additionally it has to wrap the top expression into a \lstinline{DualMonad} manually by again using a no-op multiplication (line 11):
\begin{lstlisting}
given Conversion[Double, Dual] = Dual(_, 0)
given Conversion[Int, Dual] = Dual(_, 0)

def differentiate(
    f: Dual => (Dual => DualMonad) => DualMonad, 
    x: Double
): Double =
    val xDual = Dual(x, 0)
    f(xDual) { topExpression =>
        topExpression.adjoint = 1
        topExpression * 1
    }
    xDual.adjoint
end differentiate
\end{lstlisting}
Note that the expected function \lstinline{f} again has a second argument, the continuation (\lstinline{Dual => DualMonad}) which now returns a \lstinline{DualMonad}.
Finally, we can use a for-comprehension to define \lstinline{f} to ultimately let the compiler rewrite our code into a CPS-like structure which was our goal:
\begin{lstlisting}
def f(x: Dual)(k: Dual => DualMonad): DualMonad =
    for
        y1 <- x * 2
        y2 <- x * x
        y3 <- y2 * x
        y4 <- y1 + y3
        y5 <- k(y4)
    yield y5
end f

val derivative = differentiate(f, 3)
\end{lstlisting}


\subsection{Combined Monad and Tape} \label{sec:monadAndTape}

Using for-comprehensions made the inside of a function more readable but when defining \lstinline{f} we still needed a continuation \lstinline{k} as its second parameter. This is not only ugly but makes it very hard to compose multiple functions. Instead, we want a simple way to define two functions \lstinline{f} and \lstinline{g} without seeing continuations at all and make chaining them as easy as writing \lstinline{f andThen g} as we are used to in Scala. We try to solve this by coming back to our tape based approach and combining it with our monad based one.

At first, we need our tape where we save all adjoint updates which have to happen in reverse order. This is exactly same to our first tape based implementation:
\begin{lstlisting}
var tape: Unit => Unit = _ => {}
\end{lstlisting}

Now we get back to monads. Previously we implemented \lstinline{DualMonad} ad hoc to overcome the different requirements of multiplication and addition. Those differences can be abstracted over which leads to a much cleaner and more reusable design. Essentially what we need to abstract over is the result of an parent calculation and how to update the adjoints of the child expressions using the parent adjoint. Those are both realized as members of \lstinline{DualMonad} and are called \lstinline{parent} and \lstinline{adjointsUpdater} respectively (line 1):
\begin{lstlisting}[caption={Monad using tape}, label={lst:monadTape}]
class DualMonad(val parent: Dual, val adjointsUpdater: Dual => Unit):
    def flatMap(k: Dual => DualMonad): DualMonad =
        tape = ((_: Unit) => adjointsUpdater(parent)) andThen tape
        k(parent)

    def map(k: Dual => Dual): DualMonad =
        flatMap(k andThen wrap)
end DualMonad

def wrap(dual: Dual): DualMonad = DualMonad(dual, identity)
\end{lstlisting}
\lstinline{flatMap} prepends the adjoint update operation to the tape (line 3) which is exactly what we did for our first tape implementation and then calls the continuation (line 4). \lstinline{map} is very similar to our first monad approach as it also just uses \lstinline{flatMap} after wrapping the result of \lstinline{k} (line 7). The difference is that instead of doing an no-op operation on \lstinline{Dual} we wrap it manually by passing a no-op identity function as the \lstinline{adjointUpdater} in the \lstinline{wrap} function (line 10).

We have already done the hardest task and now only have to glue the pieces together:
\begin{lstlisting}
case class Dual(x: Double, var adjoint: Double):
    def *(that: Dual): DualMonad =
        def addPartialAdjoint(
            thisOrThat: Dual,
            derivativeWrtThisOrThat: Double,
            parentAdjoint: Double
        ): Unit =
            val partialAdjoint = parentAdjoint * derivativeWrtThisOrThat
            thisOrThat.adjoint += partialAdjoint
        end addPartialAdjoint

        DualMonad(
            this.x * that.x,
            parent =>
                addPartialAdjoint(this, that.x, parent.adjoint)
                addPartialAdjoint(that, this.x, parent.adjoint)
        )
    end *

    def +(r: Dual): DualMonad = ???
end Dual
\end{lstlisting}
\lstinline{addPartialAdjoint} (line 3) is equal to previous implementations. We pass the normal result (line 13) and as usual how to update the adjoints of \lstinline{this} and \lstinline{that} to the constructor of \lstinline{DualMonad} (lines 14 to 16).

The biggest and most important change comes in the signature of \lstinline{differentiate}. The expected function \lstinline{f} uses no continuation and just expects and returns a \lstinline{DualMonad} which is exactly what we wanted:
\begin{lstlisting}
def differentiate(f: DualMonad => DualMonad)(x: Double): Double =
    tape = _ => ()
    val xDualMonad = wrap(Dual(x, 0))
    f(xDualMonad).parent.adjoint = 1
    tape(())
    xDualMonad.parent.adjoint
\end{lstlisting}
The implementation of it on the other hand is nothing special. Again, we have to remember to reset the tape (line 2). After wrapping a \lstinline{Double} into a \lstinline{Dual} and then into a \lstinline{DualMonad} (line 3) we have to call \lstinline{f} with it to do the forward pass and fill the tape (line 4). After setting the adjoint of the top expression (also line 4) we can execute the tape (and start the reverse pass) (line 5) which sets the adjoint of our argument to \lstinline{f}.

Because the expected function gets a \lstinline{DualMonad} and also returns one we can now easily chain two functions using \lstinline{andThen} (line 18) and still use for-comprehensions:
\begin{lstlisting}
def f(xM: DualMonad): DualMonad =
    for
        x <- xM
        y1 <- x * 2
        y2 <- x * x
        y3 <- y2 * x
        y4 <- y1 + y3
    yield y4
end f

def g(xM: DualMonad): DualMonad =
    for
        x <- xM
        y <- x * x
    yield y
end g

val derivative = differentiate(f andThen g)(3)
\end{lstlisting}
Note that one can interpret the first lines of the for-comprehensions (lines 3 and 13) as ``unwrapping'' the monad back into a \lstinline{Dual}.

\section{Without mutation} \label{sec:noMutation}
%\input{content/implementation/reverseMode/functional/whyFunctional.tex}
\subsection{Continuation Passing Style (CPS)}

\subsection{Monad} \label{sec:functionalMonad}

Next we want to take a look at our approach where we combined monad and tape and remove every mutation. Similar to CPS in the last section we have the mutable member \lstinline{adjoint} of \lstinline{Dual}. Our solution for it was a map which functions as an adjoint accumulator and replaces the \lstinline{adjoint} member of \lstinline{Dual}. This will prove to be useful also for this problem:
\begin{lstlisting}
type Adjoints = Map[Num, Double]
\end{lstlisting}

A completely new obstacle is the tape itself which is also mutable. To convert it into something immutable let's first reflect which purpose it served. Principally it is an recursively accumulated list of operations which have to be executed later. Nothing speaks against passing an growing function (instead of a map for example) as an accumulator of our recursive pass through an calculation. In our case these operations always specifically acted with and on adjoints of some \lstinline{Dual}. Alas \lstinline{Dual} has no \lstinline{adjoint} member anymore (and therefore was renamed into \lstinline{Num}). Its replacement is \lstinline{Adjoints} which keeps track of adjoints of every \lstinline{Num}. Consequently our ``tape accumulator'' has to act on \lstinline{Adjoints}. More specifically it has be of type \lstinline{Adjoints => Adjoints} because you update the current (passed) adjoints by adding each child's partial adjoint and returning a new updated instance of \lstinline{Adjoints}. Previously every \lstinline{DualMonad} had a member \lstinline{adjointsUpdater: Dual => Unit} which was prepended to the tape to ultimately collect every \lstinline{adjointsUpdater}. We could instead use the member \lstinline{adjointsUpdater} itself as an accumulator by changing its type. Previously \lstinline{adjointsUpdater} just signified how to update the adjoints of one subexpression. Only on the tape they were chained. Now we directly chain each \lstinline{adjointsUpdater} recursively on the fly and pass the chained \lstinline{adjointsUpdater} to the outer \lstinline{DualMonad}. Thus we are now skipping the tape entirely by moving it into an accumulator:
\begin{lstlisting}
class DualMonad(
    val parent: Num, 
    val adjointsUpdater: Adjoints => Adjoints
):
    def flatMap(k: Num => DualMonad): DualMonad =
        val outerResult = k(parent)
        DualMonad(
            outerResult.parent, 
            outerResult.adjointsUpdater andThen this.adjointsUpdater 
        )
    end flatMap

    def map(k: Num => Num): DualMonad =
        flatMap(k andThen wrap)
end DualMonad

def wrap(n: Num): DualMonad = DualMonad(n, identity)
\end{lstlisting}
\lstinline{map} follows the same logic as before. In \lstinline{flatMap} we first call the continuation to get the adjoints (and normal result) of all outer expressions (line 3). \lstinline{outerResult.adjointsUpdater} is then a tape-like construct which captures how to update all adjoints from the top expression up until the current expression. By appending \lstinline{this.adjointsUpdater} the operations are sorted from outer to inner which is the correct order we want to iterate through for the reverse pass of reverse mode differentiation. 

Implementing \lstinline{Num} is trivial. We just have to remember to always use the updated adjoint map in lines 12 to 15 because \lstinline{addPartialAdjoint} returns a fresh object instead of mutating the old one: 
\begin{lstlisting}
class Num(val x: Double):
    def *(that: Num): DualMonad =
        val parent = Num(this.x * that.x)

        def addPartialAdjoint(
            thisOrThat: Num, 
            derivativeWrtThisOrThat: Double, 
            adjoints: Adjoints
        ): Adjoints =
            val partialAdjoint = 
                adjoints(parent) * derivativeWrtThisOrThat
            val newAdjointThisOrThat = 
                adjoints(thisOrThat) + partialAdjoint
            adjoints + (thisOrThat -> newAdjointThisOrThat)
        end addPartialAdjoint

        DualMonad(
            parent,
            adjoints =>
                val adjointsWithThis = 
                    addPartialAdjoint(this, that.x, adjoints)
                val adjointsWithThat = 
                    addPartialAdjoint(that, this.x, adjointsWithThis)
                adjointsWithThat
        )
    end *

    def +(that: Num): DualMonad = ???
end Num
\end{lstlisting}

When executing \lstinline{f} in line 3 we don't get a meaningful result yet. We only have the monad of the top expression of the calculation. But this monad contains an \lstinline{adjointsUpdater} which accumulated all operations needed to update an \lstinline{Adjoints} map from scratch into our full result. We create our initial \lstinline{Adjoints} map with default value 0 and we set the top expression to 1 (line 4) just like for CPS. Finally we can call \lstinline{topMonad.adjointsUpdater} on our \lstinline{initialAdjoints} and ultimately get the adjoint of \lstinline{xM} (line 5):
\begin{lstlisting}
def differentiate(f: DualMonad => DualMonad)(x: Double): Double =
    val xM = wrap(Num(x))
    val topMonad = f(xM)
    val initialAdjoints = Map.empty.withDefaultValue(0.0).updated(topMonad.parent, 1.0)
    topMonad.adjointsUpdater(initialAdjoints)(xM.parent)
end differentiate

def f(xM: DualMonad): DualMonad =
    for
        x <- xM
        y1 <- x * 2
        y2 <- x * x
        y3 <- y2 * x
        y4 <- y1 + y3
    yield y4
end f

def g(xM: DualMonad): DualMonad =
    for
        x <- xM
        y <- x * x
    yield y
end g

val derivative = differentiate(f andThen g)(3)
\end{lstlisting}
As one can see we did not have to sacrifice much to reach an implementation without mutation. Writing and chaining functions (lines 8 to 25) are exactly the same to the previous monad implementation. The implementation itself is arguably better structured because we don't have to implement \lstinline{DualMonad} ad hoc for every operation. Removing mutation and the global variable tape variable should also make the code easier to reason about. We also prevented at least one potential bug source because one had to remember to reset the tape for every calculation which isn't a problem anymore. Furthermore we could also argue that parallel execution using a global tape could lead to a multitude of problems which we won't cover in more detail. 
\subsection{Combinatory Homomorphic Automatic Differentiation (CHAD)} \label{sec:chad}

Until now we relied on a map to accumulate all adjoints of all expressions by mapping \lstinline{Num} to \lstinline{Double} to get rid of mutation. From a functional programming standpoint this is still a compromise. We mapped specific instances of \lstinline{Num} which means we have to generate an unique ID for every object to get a functioning map. In contrast imagine we would build our map by strictly comparing values (i.e. instances of \lstinline{Num} with the same numeric value are treated as indistinguishable). Sub-expressions of our calculation with the same result value would share one adjoint which is clearly flawed. To have a purely functional implementation we have to get rid of instance comparison which means to drop the map. We also take one step back from monads and return to dual numbers:
\begin{lstlisting}
case class Dual(v: Double, variableAdjoint: Double => Double):
    def *(that: Dual): Dual =
        def variableAdjointBothSides(
            partialAdjointThis: Double, 
            partialAdjointThat: Double
        ) =
            this.variableAdjoint(partialAdjointThis) 
                + that.variableAdjoint(partialAdjointThat)

        def partialAdjointThis(parentAdjoint: Double) = 
            parentAdjoint * that.v
            
        def partialAdjointThat(parentAdjoint: Double) = 
            parentAdjoint * this.v

        Dual(
            this.v * that.v,
            parentAdjoint =>
                variableAdjointBothSides(
                    partialAdjointThis(parentAdjoint), 
                    partialAdjointThat(parentAdjoint)
                )
        )
    end *

    def +(that: Dual): Dual = ???
\end{lstlisting}
First thing to notice in line 1 is that \lstinline{Dual} now has a member named \lstinline{variableAdjoint}. It calculates an adjoint using the parent adjoint like already familiar. But the important difference (and the reason for the prefix ``variable-'') is that it does \emph{not} compute the partial adjoint of the current expression. It instead accumulates the sum of all partial adjoints of our variable (i.e. x). It ignores all partial adjoints of expressions not including x. We don't need them anyway. 

Lines 10 to 14 look very familiar because it just computes the partial adjoints of \lstinline{this} and \lstinline{that}. From line 15 on we create a new \lstinline{Dual} with a new \lstinline{variableAdjoint}.  We calculate both partial adjoints (lines 20 and 21) and then pass it to \lstinline{variableAdjointBothSides}. \lstinline{variableAdjointBothSides}'s job is to sum the partial adjoints of x from both branches but has to ensure that only relevant partial adjoints are included and for example constants are ignored. The function does in fact just sum both variable adjoints together (lines 7 and 8). So where do we ensure that only relevant partial adjoints are summed? The answer lies in how constants and the variable are instantiated/defined:
\begin{lstlisting}
def variable(v: Double): Dual = Dual(v, identity)
def const(v: Double): Dual = Dual(v, _ => 0.0)
\end{lstlisting}
The $\diff{x}{x} = 1$ partial adjoint of x is just directly its partial adjoint without further ado. This translates to the variable having \lstinline{identity} as its variable adjoint function (line 1). The variable adjoint of constants are always zero (line 2). This makes sense because constants have no meaningful partial adjoint for our variable. Effectively when \lstinline{variableAdjointBothSides} calls \lstinline{this.variableAdjoint(partialAdjointThis)} and \lstinline{this} is the variable x then it evaluates to \lstinline{partialAdjointThis}. If \lstinline{this} is a constant the call evaluates to zero. If \lstinline{this} is neither (e.g. a multiplication expression) then the previous two rules are implicitly applied for all branches of that expression recursively. This, after some consideration, rather simple design is sufficient to implement reverse mode differentiation fully functionally and without building some kind of virtual stack/tape as we did previously.

Another advantage of this approach is its simplicity for the user:
\begin{lstlisting}
given Conversion[Double, Dual] = const(_)
given Conversion[Int, Dual] = const(_)

def f(xDouble: Double): Dual =
    val x: Dual = variable(xDouble)
    2 * x + x * x * x
end f

val derivative = f(3).variableAdjoint(1)
\end{lstlisting}
After declaring the variable explicitly in line 5 (and using given conversions (lines 1 and 2) for constants) we can write \lstinline{f} simply like an usual arithmetic expression in Scala (line 6). We don't even need a \lstinline{differentiate} function because we accumulate the correct adjoint of x on the fly while calculating the result of \lstinline{f}. In previous reverse mode implementations we called \lstinline{adjoint} on x to get its adjoint which makes sense from a mathematical standpoint because we are interested in the adjoint of our variable and nothing else. In this case however we have to call \lstinline{variableAdjoint} on the result of \lstinline{f}. Ultimately we obviously still get the adjoint of x but the semantics of \lstinline{variableAdjoint} are different (as the name suggests). It can be most easily interpreted as the sum of all partial adjoints of x contained in all branches of the current expression. Therefore we must call it on the outermost expression to include all partial adjoints of x in our whole calculation.