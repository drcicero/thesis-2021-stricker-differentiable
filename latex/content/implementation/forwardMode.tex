\chapter{Forward Mode Differentiation} \label{sec:forwardMode}

Forward mode differentiation by hand is straight forward and also translates well into code by sticking to our existing knowledge about symbolic differentiation (i.e. differentiation ``by hand''). Essentially we want to go through every operation recursively and replace it with its derivative.


\section{Code replacement with macros} \label{sec:macros}
If you take the last sentence literally you could now go the long and hard way (which we did), learn metaprogramming and implement it exactly by replacing expressions with their derivative. This approach looks rather ugly at first sight but after ignoring the boilerplate (with the help of the comments on the right hand side) one can see that it just boils down to recursive pattern matching of code which works pretty intuitively in Scala 3:
\begin{lstlisting}
def d(t: Expr[Term])(using Quotes): Expr[Term] = t match
    case '{ ($l: Term) + ($r: Term) } =>    // l + r
        '{ ${ d(l) } + ${ d(r) } }          // d(l) + d(r)
    case '{ ($l: Term) * ($r: Term) } =>    // l * r
        '{ $l * ${ d(r) } + ${ d(l) } * $r }// l * d(r) + d(l) * r
    case '{ X } => '{ 1 }                   // variable
    case '{ $v: V } => '{ 0 }               // constant
\end{lstlisting}
As this was just a small experiment we leave this here without much explanation for now. In fact we wouldn't even need macros and could just write our term with algebraic data types and recursively match them at runtime to implement this approach. This would reduce the boilerplate in comparison to the macro-implementation above.


\section{Operator overloading with dual numbers}\label{sec:forwardDualNumbers}

Rewriting code that way has one problem. We loose the actual result of our formula and only get the derivative (or we would have to calculate both separately). Instead we could use a structure that consists of two values, the normal result of the operation and its derivative. Such a construct is called a "dual number" and could look like this:
\begin{lstlisting}
case class Dual(v: Double, d: Double)
\end{lstlisting}

To implement an operation on these we have to define the actual computation and additionally how to compute the derivative:
\begin{lstlisting}[caption={Dual number implementation}, label={lst:forwardDualNumber}]
case class Dual(v: Double, d: Double):
    def *(that: Dual) = Dual(
      this.v * that.v,
      this.v * that.d + this.d * that.v
    )
  
    def +(that: Dual) = Dual(
      this.v + that.v,
      this.d + that.d
    )
\end{lstlisting}
As we can see, this comes down to translating mathematic symbolic differentiation rules into code (lines 4 and 9). How to define constants and the variable (i.e. x) also comes naturally:
\begin{lstlisting}
def const(v: Double) = Dual(v, 0)
def variable(v: Double) = Dual(v, 1)
\end{lstlisting}
Differentiation of a function $f$ at $x = 3$ could then look like this:
\begin{lstlisting}
def differentiate(f: Dual => Dual)(x: Double): Double =
    val result: Dual = f(variable(x))
    result.d // result.v would be the actual result of f(x)

def f(x: Dual): Dual = const(2) * x + x * x * x

val derivative = differentiate(f)(3)  
\end{lstlisting}



\section{Static typing taken too far (Or: Let the compiler code for you)} \label{sec:matchTypes}
To explain the following approach we first have to look at match types which are essentially functions but at type level. We put one type in, look at it and decide which type to ``return''. It gets clear with the following example taken from the Scala 3 docs \cite{matchTypesScala3}. You could get the element type of any list-like type:
\begin{lstlisting}
type Elem[X] = X match
    case String => Char
    case Array[t] => t
    case Iterable[t] => t
\end{lstlisting}
By recursively matching types we can implement a differentiator purely at type level:
\begin{lstlisting}
type D[T <: Term] <: Term = T match
    case l * r => 
        l * D[r] + D[l] * r
    case l + r => 
        D[l] + D[r]
    case X => 
        V[1]
    case V[_] => 
        V[0]
\end{lstlisting}
\lstinline{T} is just a type but represents a full calculation. Furthermore notice that the \lstinline{*} and \lstinline{+} operators seen above are infix types and not methods:
\begin{lstlisting}
type +[L <: Term, R <: Term] <: Term
type *[L <: Term, R <: Term] <: Term
\end{lstlisting}
Line 7 and 9 still look suspicious. What are integers doing as a type parameter? These are compile time value types which are now supported in Scala 3:
\begin{lstlisting}
type V[C <: Double | Int] <: Term
\end{lstlisting}
\lstinline{C} is a \lstinline{Double} or \lstinline{Int} singleton value type. By using \lstinline{scala.compiletime.constValue[C]} we can extract the singleton value from a singleton type.

If we combine all functionalities described above we gain the possibility to define a function and produce its derivative entirely at type level:
\begin{lstlisting}
type F = V[2.2] * X + X * X * X
type DF = D[F] // the "symbolic" differentiation is already completed here (at compile time)

// the following is necessary to compute an actual result from the derivative (at runtime)
val df: DF = initFromType[DF] 
val result: Double = df(3)
\end{lstlisting}
Unfortunately computing a result at type level is impossible because we want to support decimal numbers. Type level calculation with integers on the other hand would be very possible and is included in Scala 3. To allow decimals we have to use a function \lstinline{initFromType} which recursively matches the provided type at compile time and constructs a function which just computes the result later at runtime. As if this init-function wasn't inelegant enough we faced another much bigger problem. The compile time when differentiating three multiplications in a row was already at about 20 s and 4 multiplications bumped it up to over 1 min (5 multiplications took too long to test). The solution to this problem was rather anticlimactic. By updating the Scala compiler to the newest version (3.0.2 as of now) compile times are now consistently at about 1 s independent of expression complexity.
